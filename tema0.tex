\section{Tema 0}

\subsection{Ejercicio 1}

Dados los vectores a=3i-2j y b=-4i+j, calcular:
\begin{enumerate}[label=\Alph*)]
  \item El vector suma y su módulo
  \item El vector diferencia y el ángulo que forma con el eje OX
  \item El vector c=2a-3b y el vector unitario que define
        la dirección y sentido de c
\end{enumerate}

\subsubsection{Apartado A}
  $$
  \overrightarrow{m} = a + b = 3i - 2j - 4i + j = -i-j
  $$

  $$
  \left\lvert m\right\rvert = \sqrt{-1^2 + -1^2} = \sqrt{2}
  $$

\subsubsection{Apartado B}
  $$
  \overrightarrow{n} = a - b = 3i - 2j + 4i - j = 7i-3j
  $$

  $$
  tan(\alpha) = \frac{n_y}{n_x} = \frac{-3}{7} = -0.4285
  $$

  $$
  \rightarrow \alpha = tan^{-1}(\frac{-3}{7}) = -23.2\degree 
  $$

\subsubsection{Apartado C}
  $$
  c = 2a - 3b = 2(3i - 2j) - 3(-4i + j)
  = 6i - 4j + 12i - 3j = 18i -7j
  $$

  $$
  \left\lvert c\right\rvert = \sqrt{18^2 + 7^2} = \sqrt(373)
  = 19.31
  $$

  $$
  \overrightarrow{u_c} = \frac{18}{19.31}i - \frac{7}{19.31}j
  = .93i - .36j
  $$

\subsection{Ejercicio 2}

Un vector tiene por origen respecto de cierto sistema
de referencia el punto O (-1, 2, 0) y de extremo
P (3, -1, 2). Calcular:
\begin{enumerate}[label=\Alph*)]
  \item Componentes del vector OP
  \item Módulo y cosenos directores
  \item Un vector unitario en la dirección de él pero de sentido contrario
\end{enumerate}

\subsubsection{Apartado A}
  $$
  \overrightarrow{OP} = \overrightarrow{P} - \overrightarrow{O} 
  = (3, -1, 2) - (-1, 2, 0) = (4, -3, 2)
  $$

\subsubsection{Apartado B}
  $$
  \left\lvert \overrightarrow{OP} \right\rvert
  = \sqrt{4^2 + 3^2 + 2^2} = \sqrt{29} = 5.38
  $$

  $$
  cos(\alpha) = \frac{4}{\sqrt{29}};
  cos(\beta) = \frac{-3}{\sqrt{29}};
  cos(\gamma) = \frac{2}{\sqrt{29}}
  $$

\subsubsection{Apartado C}
  $$
  \overrightarrow{u_{OP}}
  = (\frac{4}{\sqrt{29}}, \frac{-3}{\sqrt{29}}, \frac{2}{\sqrt{29}})
  $$

  $$
  -\overrightarrow{u_{OP}}
  = (\frac{-4}{\sqrt{29}}, \frac{3}{\sqrt{29}}, \frac{-2}{\sqrt{29}})
  $$

\subsection{Ejercicio 3}

Dados los vectores a(1, -1, 2) y b(-1, 3, 4),
calcular:
\begin{enumerate}[label=\Alph*)]
  \item El producto escalar de ambos vectores
  \item El ángulo que forman
\end{enumerate}

\subsubsection{Apartado A}
  $$
  a\cdot b = 1\cdot -1 + -1\cdot 3 + 2\cdot 4
  = 4
  $$

\subsubsection{Apartado B}
  $$
  a\cdot b = 4 = \left\lvert a\right\rvert
  \cdot \left\lvert b\right\rvert
  \cdot cos(\alpha)
  $$

  $$
  \rightarrow cos(\alpha) = \frac{4}{
    \left\lvert a\right\rvert
    \cdot \left\lvert b\right\rvert
  } = \frac{4}{\sqrt(6)\cdot \sqrt{26}}
  = \frac{4}{2\sqrt{39}} = .32
  $$

  $$
  \rightarrow \alpha = cos^{-1}(.32) = 71.32\degree
  $$

\subsection{Ejercicio 4}
Dados los vectores a=5i-2j+k y b=i-j+2k, calcular:
\begin{enumerate}[label=\Alph*)]
  \item El producto vectorial de ambos vectores
  \item El ángulo que forman
\end{enumerate}

\subsubsection{Apartado A}
  $$
  a\times b = 
  \begin{vmatrix}
    i & j & k \\
    5 &-2 & 1 \\
    1 &-1 & 2 \\
  \end{vmatrix}
  = -4i + j -5k +2k -10j +i
  = -3i -9j -3k
  $$

\subsubsection{Apartado B}
  $$
  \left\lvert a\times b\right\rvert  = \left\lvert a\right\rvert
  \cdot \left\lvert b\right\rvert
  \cdot sin(\alpha)
  $$

  $$
  \rightarrow sin(\alpha) = \frac{
    \left\lvert a\times b\right\rvert
  }{
    \left\lvert a\right\rvert 
    \cdot \left\lvert b\right\rvert 
  } = \frac{\sqrt{3^2 + 9^2 + 3^2}}{
    \sqrt{5^2 + 2^2 + 1^2}\sqrt{1^2 + 1^2 + 2^2}
  } = \frac{\sqrt{55}}{10} = .74
  $$

  $$
  \rightarrow \alpha = sin^{-1}(.74) =47.86\degree
  $$

\subsection{Ejercicio 5}
El origen de un vector es el punto A(3, -1, 2) y
su extremo B(1, 2, 1). Calcular su momento
respecto al punto C(1, 1, 2)

$$
M = \overrightarrow{CA} \times \overrightarrow{AB};
\overrightarrow{AB} = (-2, 3, -1);
\overrightarrow{CA} = (2, -2, 0);
$$

$$
M = 
\begin{vmatrix}
  i & j & k \\
  2 &-2 & 0 \\
 -2 & 3 &-1 \\
\end{vmatrix}
= 2i + 6k -4k +2j
= 2i + 2j + 2k
$$

\subsection{Ejercicio 6}
Un móvil parte de un punto con una velocidad de
110 cm/s y recorre una trayectoria rectilínea
con aceleración de -10 cm/$s^2$. Calcular el
tiempo que tardará en pasar por un punto
que dista 105 cm del punto de partida.
(Interpretar físicamente las dos soluciones
que se obtienen)

$$
y = -\frac{1}{2}10x^2 + 110x + 0
$$

Resolvamos para cuando y = 105

$$
105 = -5x^2 + 110x + 0 \rightarrow
0 = -5x^2 + 110x - 105
$$

$$
x = \frac{-110 \pm \sqrt{110^2 (-4)(-5)(-105)}}{2(-5)}
= \frac{-110 \pm \sqrt{10000}}{-10}
$$

$$
x = \frac{-110 + \sqrt{10000}}{-10} = 1
$$

$$
x = \frac{-110 - \sqrt{10000}}{-10} = 21
$$

\subsection{Ejercicio 7}
Hallar las fórmulas de un movimiento uniformemente
acelerado sabiendo que la aceleración es de 8cm/s$^2$,
que la velocidad se anula para t=3s, y que pasa
por el origen (x=0) en t=11 s.

Para la velocidad sabemos lo siguiente
$$
v = 8*3 = 24
$$

Como sabemos que la aceleración es positiva, y hace 0 a
la velocidad en algún punto del recorrido, esta tiene
que empezar negativa. Por lo tanto $v = -75$cm/s

Para el espacio recorrido, sabemos que
$$
x = \frac{1}{2}8*11^2 - 24*11 = 220
$$

Como x=0 cuando t$>$0, sabemos que x tiene que
empezar negativa. Por lo tanto, la fórmula final
es:
$$
x = \frac{1}{2}8*11 - 24*11 - 220
$$